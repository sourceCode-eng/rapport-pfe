\section{Introduction}

La mise en œuvre d’une infrastructure moderne et résiliente repose sur l’adoption d’approches déclaratives et automatisées, regroupées sous l’appellation \textit{Infrastructure as Code} (IaC). Ce paradigme vise à formaliser la définition et la gestion des ressources matérielles et virtuelles par des fichiers de configuration versionnés. L’objectif est de garantir la reproductibilité, la rapidité et la fiabilité des environnements, tout en assurant la maîtrise des coûts et la conformité aux standards de sécurité.

Ce chapitre décrit les outils et méthodes utilisés pour concevoir et automatiser l’infrastructure du projet, depuis la virtualisation des ressources jusqu’à la configuration des systèmes et la sécurisation des informations sensibles.

\section{Les outils utilisés pour l’infrastructure as code}

\subsection{Proxmox}

Proxmox Virtual Environment (Proxmox VE) est une plateforme open source de virtualisation et de gestion d’infrastructure qui combine la virtualisation basée sur des machines virtuelles (KVM) et la conteneurisation légère (LXC) dans une interface unifiée. Elle offre une solution complète pour déployer et administrer des environnements virtualisés, qu’ils soient utilisés en laboratoire, en PME ou dans des centres de données. Proxmox se distingue par sa simplicité de mise en œuvre, sa richesse fonctionnelle et sa capacité à fédérer plusieurs nœuds dans un cluster haute disponibilité.

Proxmox répond à plusieurs enjeux stratégiques : rationalisation des ressources matérielles par la mutualisation, réduction des coûts grâce à une solution libre, amélioration de la flexibilité opérationnelle et simplification de la gestion des infrastructures. Son interface web ergonomique permet de piloter l’ensemble des ressources, de planifier les sauvegardes et de superviser les performances.

D’un point de vue technique, Proxmox repose sur plusieurs composantes clés :
\begin{itemize}
    \item \textbf{KVM (Kernel-based Virtual Machine)} : moteur de virtualisation complète.
    \item \textbf{LXC (Linux Containers)} : conteneurisation système légère.
    \item \textbf{Ceph Storage} : stockage distribué intégré et hautement disponible.
    \item \textbf{Cluster Management} : fédération et basculement automatique.
    \item \textbf{Interface Web et API REST} : administration centralisée.
    \item \textbf{Sauvegardes et snapshots} : gestion de la résilience.
\end{itemize}

\textbf{Exemple d’utilisation} : déploiement d’un cluster de trois nœuds Proxmox avec stockage Ceph pour héberger des machines virtuelles critiques en haute disponibilité.

\subsection{Terraform}

Terraform est un outil open source d’Infrastructure as Code développé par HashiCorp. Il permet de définir et de provisionner des ressources complètes sous forme de code déclaratif, avec un modèle unifié pour divers fournisseurs (clouds publics, hyperviseurs privés).

Terraform répond à plusieurs enjeux : accélération du provisioning, fiabilisation des configurations et maîtrise des environnements. Il repose sur des concepts clés :
\begin{itemize}
    \item \textbf{Fichiers de configuration (HCL)} : description de l’état souhaité.
    \item \textbf{Providers} : modules d’interface avec les APIs.
    \item \textbf{State file} : enregistrement des ressources créées.
    \item \textbf{Plan d’exécution} et \textbf{apply} : gestion des changements.
    \item \textbf{Modules et workspaces} : factorisation et isolation.
\end{itemize}

\textbf{Exemple d’utilisation} : provisionner un cluster Kubernetes sur Proxmox avec un réseau et des volumes configurés.

\subsection{Cloud-init}

Cloud-init est un outil d’initialisation automatique des machines virtuelles lors du premier démarrage. Il est supporté par la plupart des plateformes cloud et hyperviseurs, et permet :
\begin{itemize}
    \item La configuration réseau.
    \item La création d’utilisateurs et de clés SSH.
    \item L’installation de paquets et le lancement de scripts.
\end{itemize}

Cloud-init contribue à standardiser les environnements et réduire les délais de mise en service.

\textbf{Exemple d’utilisation} : automatiser l’installation de Docker et la configuration réseau d’une instance Proxmox.

\subsection{Ansible}

Ansible est un outil open source d’automatisation et de configuration des systèmes, basé sur un langage déclaratif YAML et une architecture agentless (connexion SSH). Il permet de :
\begin{itemize}
    \item Définir des playbooks réutilisables.
    \item Orchestrer la configuration de plusieurs hôtes.
    \item Gérer des inventaires et des variables.
    \item Assurer la traçabilité des opérations.
\end{itemize}

\textbf{Exemple d’utilisation} : configurer des serveurs applicatifs, installer les dépendances et sécuriser les accès.

\subsection{Vault}

Vault est un gestionnaire de secrets développé par HashiCorp. Il centralise le stockage, la rotation et l’audit des informations sensibles. Les fonctionnalités principales incluent :
\begin{itemize}
    \item Chiffrement au repos et en transit.
    \item Génération dynamique de credentials.
    \item Contrôle d’accès fin (ACL).
    \item Rotation automatique des secrets.
\end{itemize}

\textbf{Exemple d’utilisation} : stockage sécurisé des tokens Kubernetes et des mots de passe base de données.

\subsection{Consul}

Consul complète Vault en apportant :
\begin{itemize}
    \item La découverte automatique des services.
    \item La supervision de leur état.
    \item Le stockage clé-valeur des configurations.
    \item Le service mesh sécurisé.
\end{itemize}

\textbf{Exemple d’utilisation} : synchroniser dynamiquement les configurations applicatives avec Consul Template.

\section{Mise en place des concepts de l'infrastructure en code}
\subsection{Preparation des secrets avec Vault}
\subsection{creation de templates de machines virtuelles}
\subsection{creation des machines virtuelles a travers Terraform}
\subsection{preparation automatique des inventaires}
\subsection{configuration automatique des machines virtuelles avec Ansible}
\section{Outils de reseau exposition des services et sécurité}

\subsection{pfSense}

pfSense est une solution open source de pare-feu et de routage. Elle permet :
\begin{itemize}
    \item La définition de règles de filtrage réseau.
    \item La gestion de VPN et la segmentation des VLAN.
    \item La supervision du trafic.
\end{itemize}
\subsection{reseau de k8s}
\subsection{MetalLB}

MetalLB est un Load Balancer pour Kubernetes on-premise :
\begin{itemize}
    \item Attribution d’adresses IP virtuelles.
    \item Distribution du trafic vers les pods.
\end{itemize}

\textbf{Exemple d’utilisation} : exposer des services applicatifs Kubernetes en haute disponibilité.
\section{Mise en places des services de reseau}
\subsection{configuration automatique de pfSense avec Ansible}
\subsection{configuration de nginx avec Ansible}
\subsection{usage de vault pour la gestion des secrets}
\section{Synthèse}

La combinaison d’outils tels que Proxmox, Terraform, Ansible, Vault et pfSense a permis de construire une infrastructure automatisée, sécurisée et reproductible. Cette approche s’inscrit dans la démarche Infrastructure as Code, garantissant un haut niveau de cohérence et facilitant les évolutions futures.

