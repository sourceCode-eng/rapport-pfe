\section{Introduction}

% Présente le contexte de la réalisation : comment le projet s'est déroulé, les contraintes, le planning général.

\section{Objectifs du projet}

% Décris les objectifs concrets du projet, les résultats attendus, la valeur ajoutée.

\section{Architecture du projet}

% Introduction générale sur l'architecture globale.
% Tu peux faire un schéma d'architecture si tu le souhaites.

\subsection{Infrastructure as Code avec Terraform}

% Explique comment Terraform a été utilisé pour provisionner l'infrastructure.
% Décris les choix techniques, les modules, etc.

\subsection{Configuration automatique avec Ansible}

% Présente la configuration des serveurs, les rôles Ansible, l'automatisation.

\subsection{GitOps avec Argo CD}

% Explique comment tu as utilisé Argo CD pour gérer les déploiements.

\subsection{Monitoring et observabilité avec Grafana, Prometheus, Loki et Tempo}

% Décris la mise en place du monitoring et de la collecte de logs.

\subsection{Stockage distribué avec Longhorn}

% Présente l'implémentation du stockage persistant.

\subsection{Gestion des secrets avec Vault}

% Explique comment Vault a été intégré pour gérer les secrets.

\subsection{Mise en place de services internes}

% Présente les services internes installés et leur rôle.

\subsection{Sécurité et gestion de pare-feu avec pfSense}

% Décris la configuration du pare-feu, les règles mises en place.

\subsection{Mise en place des services de test et de staging}

% Explique comment les environnements de test/staging ont été déployés.

\subsection{Mise en place de la CI/CD avec GitLab CI}

% Décris les pipelines, les processus de build et déploiement automatisés.

\subsection{Mise en place des services de production pour les clients}

% Présente la finalisation des environnements de production, les validations.

