\section{Presentation des outils GitOps}
\subsection{Argo CD}

Argo CD est un outil open source de déploiement continu (CD) natif Kubernetes, conçu pour mettre en œuvre les pratiques GitOps. Il permet de synchroniser l’état désiré des applications, défini dans un dépôt Git, avec l’état effectif du cluster Kubernetes. En automatisant la gestion et le déploiement des manifestes, Argo CD apporte cohérence, traçabilité et résilience aux environnements cloud-native.

%point de vue metier
Argo CD répond à plusieurs enjeux stratégiques  : fiabiliser les déploiements, réduire le temps de mise en production, renforcer la traçabilité et limiter les erreurs humaines. Il offre un modèle déclaratif et auditable, conforme aux exigences de sécurité et de conformité des organisations modernes. En industrialisant le GitOps, Argo CD contribue à accélérer l’innovation tout en garantissant la stabilité des systèmes.

%point de vue logique et technique
, Argo CD s’appuie sur plusieurs composants clés  :
\begin{itemize}
	\item \textbf{Le dépôt Git}  : source unique de vérité contenant les manifestes Kubernetes (YAML) ou les définitions Kustomize/Helm.
	\item \textbf{Le contrôleur Argo CD}  : composant qui surveille les différences entre l’état souhaité (Git) et l’état réel du cluster.
	\item \textbf{L’API Server et l’interface Web}  : couche d’administration et de visualisation centralisée des applications et des synchronisations.
	\item \textbf{Les applications}  : objets Kubernetes représentant l’état désiré d’un ensemble de ressources.
	\item \textbf{Les stratégies de synchronisation}  : modes automatique ou manuel permettant de contrôler les mises à jour.
\end{itemize}

Argo CD offre un modèle de sécurité avancé, intégrant la gestion fine des permissions (RBAC), le support du SSO (OAuth2, OIDC), le chiffrement des secrets et des validations automatiques des changements.

\textbf{Exemples et cas d’usage} :
\begin{itemize}
	\item Déployer automatiquement une application Helm versionnée depuis un dépôt Git centralisé.
	\item Gérer des environnements multiples (dev, staging, production) avec des dossiers ou des branches distinctes.
	\item Appliquer des politiques de synchronisation automatique avec validation de signature Git.
	\item Visualiser les différences entre l’état courant et l’état cible et lancer un déploiement manuel.
	\item Auditer l’historique des déploiements et des changements appliqués au cluster.
\end{itemize}

\textbf{Avantages principaux} :
\begin{itemize}
	\item Mise en œuvre native du GitOps et centralisation de la configuration déclarative.
	\item Traçabilité et auditabilité complètes des changements.
	\item Intégration fluide avec Helm, Kustomize, Jsonnet et plain YAML.
	\item Réduction du risque d’erreurs grâce au contrôle automatique des dérives d’état.
	\item Interface Web ergonomique et API REST.
	\item Sécurité renforcée avec RBAC et chiffrement des secrets.
\end{itemize}

En synthèse, Argo CD est une solution stratégique pour l’automatisation et la fiabilisation des déploiements Kubernetes. Il contribue à instaurer des workflows GitOps robustes, cohérents et évolutifs, adaptés aux exigences opérationnelles des entreprises modernes.

\textbf{Références suggérées} :
\begin{itemize}
	\item Argo CD Documentation – \url{https://argo-cd.readthedocs.io/}
	\item Argo CD GitHub Repository – \url{https://github.com/argoproj/argo-cd}
	\item GitOps Principles – \url{https://www.gitops.tech/}
	\item CNCF Argo Project – \url{https://www.cncf.io/projects/argo/}
	\item Helm Documentation – \url{https://helm.sh/docs/}
\end{itemize}

\subsection{Helm}

Helm est un gestionnaire de packages Kubernetes qui permet de décrire des applications sous forme de charts. Chaque chart contient des modèles de manifestes et des fichiers de valeurs qui définissent les paramètres de l’application. Helm est particulièrement adapté lorsque :
\begin{itemize}
	\item Les applications déployées nécessitent de nombreuses options configurables.
	\item Il est souhaité de réutiliser des packages officiels ou communautaires (par exemple nginx, prometheus).
	\item Les équipes ont besoin de versionner et de publier des applications sous forme de packages standardisés.
\end{itemize}
\textbf{Exemples de cas d’usage} :
\begin{itemize}
	\item Déploiement d’un cluster Prometheus avec des paramètres personnalisés.
	\item Installation d’un Ingress Controller en adaptant les valeurs selon l’environnement.
\end{itemize}
\textbf{Exécution} : Lorsqu’Argo CD synchronise une application déclarée comme Helm dans le dépôt Git, il exécute le rendu du chart (commande helm template) avant d’appliquer les ressources générées dans le cluster.

\subsection{Kustomize}

Kustomize est un outil natif Kubernetes de composition et de personnalisation des manifestes YAML. Il fonctionne en surchargeant des ressources de base avec des patches et des overlays. Kustomize est adapté lorsque :
\begin{itemize}
	\item L’application ne nécessite pas un système de packaging complet.
	\item Les environnements (dev, recette, prod) partagent une base commune mais nécessitent des ajustements ciblés.
	\item Il est important de conserver des manifestes Kubernetes purs et lisibles.
\end{itemize}
\textbf{Exemples de cas d’usage} :
\begin{itemize}
	\item Appliquer des replicas différents selon l’environnement.
	\item Modifier les variables d’environnement ou les annotations.
\end{itemize}
\textbf{Exécution} : Argo CD supporte nativement Kustomize. Lors de la synchronisation, le contrôleur applique kustomize build pour générer les manifestes avant de les appliquer au cluster.

\paragraph{Intégration avec Argo CD}

Argo CD offre un support natif pour Helm et Kustomize. Lors de la définition d’une application, il suffit de préciser le type de source (Helm, Kustomize ou Directory). Le processus de rendu est alors entièrement automatisé :
\begin{itemize}
	\item Le contrôleur Argo CD surveille le dépôt Git et détecte les modifications.
	\item Il exécute le rendu des manifestes via Helm ou Kustomize.
	\item Les ressources générées sont comparées à l’état courant du cluster.
	\item Les différences sont appliquées automatiquement ou manuellement selon la stratégie choisie.
\end{itemize}

\paragraph{Bénéfices principaux}

Le recours à Helm et Kustomize apporte plusieurs avantages :
\begin{itemize}
	\item Réduction de la duplication des manifestes entre environnements.
	\item Simplification de la gestion des configurations dynamiques.
	\item Meilleure lisibilité et maintenabilité des définitions Kubernetes.
	\item Standardisation des déploiements grâce aux charts officiels ou internes.
\end{itemize}

\section{Mise en œuvre du modèle GitOps}

Le modèle GitOps vise à centraliser la définition de l’infrastructure et des applications dans des dépôts Git versionnés, en s’appuyant sur un opérateur qui applique automatiquement l’état souhaité dans le cluster Kubernetes.
Dans ce projet, l’outil \textbf{Argo CD} a été retenu pour assurer ce rôle.
La démarche GitOps permet d’améliorer la traçabilité, la cohérence et l’automatisation des déploiements.

\subsection{Préparation des manifestes des outils internes}

Avant l’installation d’Argo CD, les manifestes Kubernetes décrivant les composants internes nécessaires au bon fonctionnement de la plateforme ont été préparés.
Ces manifestes incluent :
\begin{itemize}
	\item Les configurations des namespaces réservés (par exemple \texttt{argocd}, \texttt{monitoring}, \texttt{tools}).
	\item Les déploiements de services annexes tels que les opérateurs de sauvegarde et les contrôleurs réseau.
	\item Les configurations des ressources communes (ConfigMaps, Secrets, RBAC).
\end{itemize}

L’ensemble de ces manifestes est versionné dans un dépôt Git dédié à l’infrastructure, garantissant une source unique de vérité et la possibilité de reconstruire intégralement l’environnement.

\subsection{Installation d’Argo CD}

L’installation d’Argo CD a été réalisée via l’application des manifestes officiels fournis par le projet.
Le processus s’effectue en deux étapes principales :
\begin{itemize}
	\item Création du namespace dédié (\texttt{argocd}).
	\item Application du manifest d’installation complet :
	      \begin{verbatim}
kubectl apply -n argocd -f https://raw.githubusercontent.com/argoproj/argo-cd/stable/manifests/install.yaml
\end{verbatim}
\end{itemize}

Après l’installation, les pods principaux (API server, repo server, application controller et dex) sont déployés automatiquement.
L’interface web d’Argo CD permet de superviser les applications GitOps et leur état de synchronisation.

\subsection{Configuration de l’authentification}

La sécurisation de l’accès à Argo CD est essentielle.
Les mesures suivantes ont été mises en place :
\begin{itemize}
	\item Activation de l’authentification via Dex avec un connecteur LDAP, permettant une intégration avec l’annuaire interne.
	\item Création de rôles et de policies RBAC pour définir des droits différenciés selon les équipes (lecture seule, modification, administration).
	\item Rotation automatique des tokens d’accès.
	\item Activation de TLS pour sécuriser les communications avec l’interface web.
\end{itemize}

Ces mécanismes garantissent que seuls les utilisateurs autorisés peuvent interagir avec les ressources et déclencher des déploiements.

\subsection{Configuration des synchronisations}

La synchronisation automatique entre l’état déclaré dans Git et l’état réel du cluster est un principe fondamental de GitOps.
Argo CD a été configuré avec les paramètres suivants :
\begin{itemize}
	\item Mode de synchronisation automatique (\texttt{auto-sync}) activé sur les applications critiques.
	\item Validation stricte des manifests avant application.
	\item Prise en charge des stratégies de \textit{pruning} pour supprimer les ressources obsolètes.
	\item Notification par webhook et alerting en cas d’écart détecté entre l’état souhaité et l’état courant.
\end{itemize}

Cette configuration permet de garantir que le cluster converge toujours vers l’état décrit dans les dépôts Git et de détecter les modifications manuelles non autorisées.

\subsection{Préparation des manifestes des applications développées par Oneex pour des environnements différents}

Les applications développées par Oneex ont été déployées sur plusieurs environnements (développement, recette, production).
Pour assurer la cohérence et l’adaptabilité, les manifestes Kubernetes ont été préparés selon les principes suivants :
\begin{itemize}
	\item Utilisation de \texttt{kustomize} pour générer des variantes par environnement (par exemple configuration des replicas, des ressources et des variables d’environnement).
	\item Définition de ConfigMaps et de Secrets séparés selon les contextes.
	\item Structuration des dépôts Git avec des arborescences claires par application et par environnement.
	\item Mise en place de règles de validation continue (linting et contrôle de schéma) avant validation des commits.
\end{itemize}

Cette approche permet de disposer d’un processus de déploiement uniforme, de simplifier la maintenance et de garantir que chaque environnement est conforme aux spécifications attendues.
