\thispagestyle{mainmatter}

\section{Objectifs et contexte du projet}
\subsection{Contexte et enjeux du projet}

Dans un environnement où les systèmes d'information deviennent de plus en plus complexes et interconnectés, la gestion manuelle des infrastructures techniques pose de nombreux défis. Les entreprises doivent faire face à des exigences croissantes en matière de sécurité, de disponibilité et de performance, tout en cherchant à optimiser leurs coûts et à réduire les délais de mise en production. Cette évolution rend la gestion des infrastructures non seulement complexe, mais parfois inefficace, voire impossible à grande échelle.

Dans ce contexte, l'automatisation des processus d'infrastructure et l'adoption de solutions DevOps sont devenues des priorités stratégiques. La maîtrise de l'infrastructure et des processus de déploiement passe alors d'un luxe à une nécessité afin de soutenir rapidement et efficacement la croissance continue des besoins de l'entreprise.

Cependant, automatiser le déploiement des services ne suffit plus. Il est essentiel de garantir à tout moment leur bon fonctionnement grâce à des mécanismes de supervision et de contrôle rigoureux. La mise en place de solutions de \emph{monitoring}, de \emph{logging} et de \emph{tracing} permet de disposer d'une visibilité complète sur l'état des systèmes, d'anticiper les incidents et de réagir rapidement en cas d'anomalie. Ces dispositifs contribuent également à renforcer la traçabilité et à répondre aux impératifs de conformité réglementaire.

En parallèle, le renforcement de la cybersécurité constitue un enjeu majeur. La multiplication des points d'entrée et l'interconnexion croissante des services exposent l'infrastructure à de nouvelles menaces qu'il convient de prévenir et de détecter de manière proactive.

Ce mémoire s'inscrit dans cette dynamique, avec pour objectif principal de concevoir et mettre en place une solution automatisée, sécurisée et résiliente permettant de déployer, superviser et maintenir l'infrastructure technique de l'entreprise Oneex. Le projet vise à répondre aux besoins opérationnels croissants, à réduire les erreurs manuelles et à garantir un haut niveau de qualité de service et de transparence, tout en respectant les contraintes strictes de sécurité et de conformité réglementaire.

\subsection{Scénarios de référence et hypothèses de travail}

La conception d’une solution d’automatisation et de sécurisation des infrastructures ne peut se limiter à une approche purement théorique. Elle doit s’appuyer sur l’analyse de situations concrètes, issues de l’expérience opérationnelle et des incidents effectivement rencontrés dans des environnements comparables à celui de l’entreprise Oneex.
Dans cette perspective, plusieurs scénarios représentatifs ont été identifiés afin d’illustrer les risques, les besoins et les priorités du projet.
Ces scénarios reflètent des problématiques récurrentes observées lors de la gestion quotidienne des systèmes : cohérence entre les environnements, gestion des défaillances matérielles ou logicielles, continuité de service, sécurité des données et qualité de vie au travail des équipes techniques. La prise en compte de ces aspects vise à construire un environnement plus fiable, plus transparent et moins générateur de stress pour les équipes Dev et Ops. La suite de ce chapitre présente ces scénarios, qui serviront de base aux choix techniques et organisationnels retenus pour la solution cible.

\paragraph{\textbf{Divergences entre environnements}}

Dans de nombreux contextes, la configuration manuelle des serveurs opérée par plusieurs équipes successives conduit progressivement à des écarts significatifs entre les environnements de développement, de test et de production. Ces divergences concernent notamment :

\begin{itemize}
	\item Les versions des systèmes d'exploitation, des librairies et des dépendances logicielles.
	\item Les paramètres réseau tels que l'ouverture de ports ou l'attribution d'adresses IP.
	\item La définition des règles de sécurité (droits d'accès, politiques de pare-feu).
\end{itemize}

Ces différences génèrent fréquemment des dysfonctionnements applicatifs lors des bascules d'environnement, compliquent la reproduction des incidents constatés et placent l’équipe opérations sous forte pression en raison de délais de résolution prolongés. Si elles provoquent des interruptions de service, ces situations peuvent également nuire à l’image de sérieux et de fiabilité de l’entreprise auprès de ses clients et partenaires.

\paragraph{\textbf{Gestion hétérogène des secrets}}

Il est fréquent de constater l’absence de processus unifié de gestion des informations sensibles (identifiants, clés d’API, certificats). Ces éléments sont souvent stockés et partagés de façon informelle, par exemple :

\begin{itemize}
	\item Sous forme de fichiers non chiffrés sur les postes individuels.
	\item Par échange de courriels non sécurisés.
	\item Par messagerie instantanée, sans traçabilité ni archivage structuré.
\end{itemize}

Une telle organisation expose les infrastructures à des risques accrus de fuite d’informations critiques et complique la gestion opérationnelle lors des renouvellements ou révocations des secrets. Elle entraîne également un déficit de traçabilité sur les accès et les modifications, et augmente la probabilité de perdre définitivement certains secrets sensibles. Ces défaillances peuvent aller jusqu’à bloquer totalement l’accès aux ressources stratégiques de l’entreprise, y compris lors d’incidents pourtant courants, comme le vol ou la panne d’un poste administrateur.

\paragraph{\textbf{Déficit d'observabilité}}

Il est également courant qu’aucun dispositif unifié de supervision et de journalisation n’ait été mis en place. Dans ce contexte, plusieurs limitations apparaissent :

\begin{itemize}
	\item L’absence de collecte systématique des métriques de performance.
	\item La dispersion des journaux applicatifs sur des serveurs multiples, sans agrégation centralisée.
	\item Le manque de mécanismes de corrélation des événements entre composants.
\end{itemize}

Un tel déficit d’observabilité réduit considérablement la capacité à détecter précocement les anomalies, à diagnostiquer efficacement les causes racines et à mesurer le respect des engagements de qualité de service.

\paragraph{\textbf{Processus de déploiement manuel et long}}

Dans certaines organisations, le déploiement d’une infrastructure repose encore sur des opérations entièrement manuelles qui sont souvent réalisées en dehors des horaires de service (la nuit ou pendant les week-ends) afin de limiter l’impact sur les utilisateurs.. La création d’un nouvel environnement peut alors nécessiter plusieurs jours de travail successif :

\begin{enumerate}
	\item Préparation et allocation des ressources matérielles ou virtuelles.
	\item Installation des systèmes d’exploitation et des dépendances logicielles.
	\item Paramétrage des droits d’accès et des configurations de sécurité.
	\item Vérification manuelle de la conformité et du bon fonctionnement des services.
\end{enumerate}

Ces processus induisent des délais importants, une faible reproductibilité et une exposition accrue aux erreurs humaines.

\paragraph{\textbf{Risques opérationnels associés}}

Ces scénarios convergent vers un ensemble de risques concrets et bien identifiés :

\begin{itemize}
	\item L’allongement des délais de livraison et la perte d’agilité opérationnelle.
	\item L’augmentation de la surface d’attaque due à des configurations hétérogènes et non sécurisées.
	\item La probabilité accrue d’erreurs de configuration.
	\item La difficulté à garantir la sécurité des environnements et la confidentialité des données sensibles.
	\item L’impossibilité de disposer d’une vision globale et en temps réel de l’état de l’infrastructure.
	\item Des temps de diagnostic et de réaction allongés, rendant plus difficile la résolution rapide des problèmes critiques.
\end{itemize}

Ces constats renforcent l’intérêt d’intégrer dès la conception des mécanismes de \emph{monitoring}, de \emph{logging} et de \emph{tracing} afin de renforcer la transparence et la capacité de réaction face aux incidents.

\paragraph{\textbf{Justification des orientations retenues}}

L’analyse de ces situations et des risques associés conduit à considérer comme prioritaires les axes suivants :

\begin{itemize}
	\item L’automatisation des processus de déploiement et de configuration, pour réduire les délais et améliorer la cohérence.
	\item La centralisation et la sécurisation de la gestion des secrets et des accès, afin de prévenir les fuites d’informations sensibles.
	\item Lorsque cela est possible, l’utilisation de mécanismes de génération et de rotation automatique de secrets temporaires (par exemple, des credentials ou des clés TLS via des solutions de type \emph{Vault}).
	\item La mise en place d’une gestion stricte des droits d’accès, fondée sur le principe du moindre privilège et les bonnes pratiques du \emph{Zero Trust}.
	\item L’intégration d’outils d’observabilité pour assurer un suivi continu et une traçabilité complète des opérations.
	\item Le renforcement des contrôles de sécurité et la conformité avec les normes en vigueur.
\end{itemize}

Ces orientations constituent le socle sur lequel s'appuie le projet présenté dans ce mémoire.

\begin{longtable}{|p{5cm}|p{5cm}|p{5cm}|}
	\caption{Synthèse des cas d'usage avant et après la mise en place de la solution} \label{tab:avant_apres}                                                                                           \\
	\hline
	\textbf{Aspect}                                                                                                              & \textbf{Situation initiale} & \textbf{Situation après mise en place} \\
	\hline
	\endfirsthead
	\hline
	\textbf{Aspect}                                                                                                              & \textbf{Situation initiale} & \textbf{Situation après mise en place} \\
	\hline
	\endhead
	\hline
	\multicolumn{3}{r}{\textit{Suite en page suivante}}
	\endfoot
	\hline
	\endlastfoot
	\textbf{Cohérence des environnements}                                                                                        &
	Configurations manuelles et divergentes entre développement, test et production, générant des écarts difficiles à maîtriser. &
	Automatisation des déploiements garantissant la standardisation et la reproductibilité des environnements.                                                                                          \\
	\hline
	\textbf{Gestion des secrets}                                                                                                 &
	Stockage informel et non sécurisé des identifiants, échanges par email ou messagerie instantanée.                            &
	Centralisation sécurisée des secrets avec rotation automatique et contrôle des accès.                                                                                                               \\
	\hline
	\textbf{Observabilité}                                                                                                       &
	Absence d’agrégation centralisée des journaux et des métriques, difficultés de diagnostic.                                   &
	Mise en place d’outils de supervision, de journalisation et de traçabilité unifiés.                                                                                                                 \\
	\hline
	\textbf{Déploiement des infrastructures}                                                                                     &
	Processus manuel long (plusieurs jours), faible reproductibilité et risque élevé d’erreurs humaines.                         &
	Automatisation complète permettant des déploiements rapides et contrôlés.                                                                                                                           \\
	\hline
	\textbf{Sécurité opérationnelle}                                                                                             &
	Droits d’accès gérés de façon hétérogène et peu contrôlée, exposition accrue aux attaques.                                   &
	Application du principe du moindre privilège et d’une approche Zero Trust, contrôles renforcés.                                                                                                     \\
	\hline
	\textbf{Temps de réaction aux incidents}                                                                                     &
	Diagnostic lent et difficile en cas d’anomalie ou d’incident critique.                                                       &
	Observabilité en temps réel et capacité d’investigation rapide grâce à la centralisation des informations.                                                                                          \\
	\hline
\end{longtable}

\subsection{Les besoins fonctionnels}

Les besoins fonctionnels décrivent l'ensemble des fonctionnalités attendues de la solution envisagée, ainsi que les objectifs opérationnels qui en découlent. Ils visent à garantir la cohérence, la sécurité, la traçabilité et la résilience de l'infrastructure et des applications. Ces besoins peuvent être regroupés autour de plusieurs axes principaux :

\paragraph{\textbf{Provisioning et configuration des ressources}}

\begin{itemize}
	\item \textbf{Automatisation du provisioning des ressources} : permettre la création, la configuration et la suppression des composants d'infrastructure de manière déclarative et reproductible, afin de réduire les délais et d'éviter les interventions manuelles.
	\item \textbf{Gestion centralisée et cohérente des configurations} : mettre en œuvre un mécanisme d'orchestration permettant d'installer les dépendances logicielles, d'appliquer les paramétrages requis et de maintenir l'uniformité entre les différents environnements.
\end{itemize}

\paragraph{\textbf{Déploiement et mise à jour des applications}}

\begin{itemize}
	\item \textbf{Déploiement applicatif automatisé et contrôlé} : intégrer un processus déclenchant les déploiements depuis des référentiels versionnés et assurant la synchronisation permanente entre le code source et les environnements cibles.
\end{itemize}

\paragraph{\textbf{Supervision et observabilité}}

\begin{itemize}
	\item \textbf{Supervision proactive et alertes en temps réel} : disposer d'un système de surveillance permettant de collecter les métriques de performance, de visualiser l'état des services et de générer des alertes en cas d'incident.
	\item \textbf{Détection précoce des anomalies} : mettre en place des mécanismes d'analyse continue et d'identification des écarts de comportement afin d'anticiper les incidents et de réduire leur impact.
	\item \textbf{Journalisation centralisée} : assurer la collecte, le stockage et la consultation unifiée des journaux système et applicatifs.
\end{itemize}

\paragraph{\textbf{Sécurité et gestion des accès}}

\begin{itemize}
	\item \textbf{Gestion sécurisée et dynamique des secrets} : intégrer un système centralisé de stockage, de chiffrement et de distribution des informations sensibles, avec des mécanismes de rotation automatique et de durée de vie limitée des secrets lorsque cela est possible.
	\item \textbf{Séparation stricte des environnements} : organiser l'infrastructure en environnements distincts (développement, test, pré-production, production) afin de garantir leur isolation et de limiter les risques de contamination croisée.
\end{itemize}

\paragraph{\textbf{Résilience et continuité de service}}

\begin{itemize}
	\item \textbf{Correction automatique des incidents et des défaillances} : prévoir des processus d'auto-remédiation capables de restaurer l'état nominal des services, par exemple par le redémarrage ou le reprovisionnement automatisé des ressources en cas de panne.
\end{itemize}

\paragraph{\textbf{Interface de pilotage}}

\begin{itemize}
	\item \textbf{Interface unifiée d'administration} : proposer une interface utilisateur et/ou une API permettant d'interagir avec la plateforme de manière sécurisée et traçable.
\end{itemize}

Ces besoins fonctionnels constituent la base de la solution à concevoir, en intégrant les outils et les pratiques DevOps adaptés pour répondre aux enjeux opérationnels et réglementaires de l'entreprise.

\paragraph{\textbf{Les besoins non fonctionnels}}

Les besoins non fonctionnels définissent les critères de qualité, de performance, de sécurité et de conformité que la solution doit respecter de manière transversale. Ces exigences sont essentielles pour garantir la fiabilité, la pérennité et la valeur ajoutée de la plateforme. Elles peuvent être regroupées selon plusieurs dimensions complémentaires.

\paragraph{\textbf{Qualité de service et performance}}

\begin{itemize}
	\item \textbf{Haute disponibilité} : garantir un taux de disponibilité supérieur à 99,9\,\% pour les services critiques, en prévoyant des mécanismes de redondance, de bascule automatique et de tolérance aux pannes.
	\item \textbf{Performance} : assurer des temps de réponse optimaux et constants, y compris en période de forte charge, afin de préserver la qualité d'expérience des utilisateurs et le respect des engagements contractuels (SLA).
	\item \textbf{Scalabilité} : permettre une montée en charge fluide et progressive de l'infrastructure, que ce soit en termes de volume de données, de nombre d'utilisateurs ou de capacités de traitement.
	\item \textbf{Réduction du temps de mise en production} : optimiser les processus afin de diminuer significativement les délais nécessaires au déploiement de nouvelles fonctionnalités ou de correctifs.
\end{itemize}

\paragraph{\textbf{Sécurité et protection des données}}

\begin{itemize}
	\item \textbf{Sécurité renforcée} : garantir la protection des données sensibles, la confidentialité des échanges et la résilience face aux attaques, en appliquant les principes de \emph{security by design} et en intégrant les contrôles de sécurité dès la conception.
	\item \textbf{Gestion stricte des droits d'accès} : appliquer le principe du moindre privilège, segmenter les privilèges et mettre en œuvre des mécanismes de contrôle d'accès granulaires et auditables, conformément aux approches de type Zero Trust.
	\item \textbf{Réduction du mouvement latéral} : cloisonner les composants et les services pour limiter les possibilités de propagation en cas de compromission, et empêcher qu’un attaquant ayant compromis un service puisse accéder aux autres ressources du système.
	\item \textbf{Traçabilité et auditabilité} : conserver un historique détaillé, horodaté et inviolable de toutes les opérations critiques, des changements de configuration et des actions administratives.
	\item \textbf{Audits automatiques de conformité et d'intégrité} : générer périodiquement des rapports permettant de vérifier la cohérence des configurations, la robustesse des mécanismes de sécurité et le respect des politiques internes et réglementaires.
\end{itemize}

\paragraph{\textbf{Évolutivité et maintenabilité}}

\begin{itemize}
	\item \textbf{Maintenabilité} : faciliter l'application des mises à jour logicielles, l'évolution des configurations et l'intégration de nouvelles fonctionnalités, tout en minimisant les interruptions de service.
	\item \textbf{Source unique de vérité (Single Source of Truth)} : centraliser et versionner l'ensemble des configurations, des états d'infrastructure et de la documentation technique dans un référentiel unique, fiable et auditable.
	\item \textbf{Respect du principe DRY (Don’t Repeat Yourself)} : structurer les configurations et les processus de manière modulaire et réutilisable, afin d'éviter les duplications inutiles, de garantir la cohérence et de réduire le risque d'erreurs lors des modifications ou évolutions.
\end{itemize}

\paragraph{\textbf{Conformité réglementaire et normes applicables}}

\begin{itemize}
	\item \textbf{Conformité réglementaire} : respecter les obligations légales et les standards sectoriels en vigueur notamment RGPD, ANSII mais egalement ISO 27001, NIS2, etc. , ainsi que les exigences spécifiques à l'activité et aux données traitées.
\end{itemize}