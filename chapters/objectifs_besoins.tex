\section{Introduction}

% Ici, tu expliques le contexte général des besoins :
% Pourquoi ce mémoire ?
% Quel est le périmètre général du projet ?
% Quelle problématique répond-il ?

\section{Cas et exemples}

% Présente ici des exemples concrets ou des situations rencontrées dans l'entreprise 
% qui justifient la mise en place du projet.
% Tu peux donner des cas d'usage, des incidents récurrents, etc.

\section{Les besoins fonctionnels}

% Liste et décris ici les besoins fonctionnels (les fonctionnalités attendues).
% Par exemple :
% - Automatisation des déploiements
% - Centralisation de la gestion des secrets
% - Supervision et alertes
% - ...
% N'hésite pas à utiliser un itemize :

\begin{itemize}
    \item Automatisation des processus de provisioning.
    \item Gestion centralisée des configurations.
    \item Mise en place d'un monitoring complet.
    \item Sécurisation des accès aux services.
\end{itemize}

\section{Les besoins non fonctionnels}

% Ici, tu décris les contraintes de performance, sécurité, disponibilité, maintenabilité, etc.
% Par exemple :
% - Disponibilité 99,9%
% - Respect des normes de sécurité
% - Simplicité de maintenance
% - Documentation claire
% Tu peux aussi faire un itemize si c'est plus lisible.

\begin{itemize}
    \item Haute disponibilité des services critiques.
    \item Sécurisation des données sensibles.
    \item Scalabilité pour accueillir de nouveaux projets.
    \item Réduction du temps moyen de déploiement.
\end{itemize}
