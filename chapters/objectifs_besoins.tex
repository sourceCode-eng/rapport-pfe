\section{Introduction}

Dans un environnement où les systèmes d'information deviennent de plus en plus complexes et interconnectés, la maîtrise de l'infrastructure et des processus de déploiement est un enjeu majeur. Ce mémoire s'inscrit dans cette dynamique, avec pour objectif principal de concevoir et mettre en place une solution automatisée et sécurisée permettant de déployer, superviser et maintenir l'infrastructure technique de l'entreprise Oneex. 

Le projet vise à répondre aux besoins opérationnels croissants, à réduire les erreurs manuelles et à garantir un haut niveau de qualité de service, tout en respectant les contraintes de sécurité et de conformité réglementaire.

\section{Cas et exemples}

Plusieurs situations rencontrées chez Oneex ont mis en évidence la nécessité d'une solution d'automatisation et de gestion centralisée de l'infrastructure. Parmi les cas récurrents :

\begin{itemize}
    \item \textbf{Problèmes de cohérence des environnements} : la configuration manuelle des serveurs engendrait des divergences entre les environnements de développement, de test et de production.
    \item \textbf{Difficultés de gestion des secrets} : le stockage et le partage des identifiants, clés d'API et certificats étaient réalisés de façon disparate, augmentant les risques de fuite.
    \item \textbf{Manque de visibilité} : l'absence d'outils de supervision unifiés complexifiait le suivi de l'état des services et la détection des incidents.
    \item \textbf{Temps de déploiement élevé} : chaque mise en place d'une infrastructure nécessitait plusieurs jours de préparation et de validation.
\end{itemize}

Ces constats ont motivé la définition d'un projet structuré et ambitieux, articulé autour de l'automatisation, de la sécurité et de l'observabilité.

\section{Les besoins fonctionnels}

Les besoins fonctionnels définissent les fonctionnalités attendues de la solution mise en œuvre. Ils se déclinent comme suit :

\begin{itemize}
    \item \textbf{Automatisation du provisioning} : création et configuration des machines virtuelles via des outils d'Infrastructure as Code (Terraform).
    \item \textbf{Gestion centralisée des configurations} : mise en place d'Ansible pour orchestrer l'installation des paquets et le paramétrage des systèmes.
    \item \textbf{Déploiement applicatif automatisé} : utilisation d'un processus GitOps avec Argo CD pour garantir la cohérence des déploiements.
    \item \textbf{Supervision et alertes} : intégration de Prometheus et Grafana pour la collecte et l'affichage des métriques.
    \item \textbf{Gestion sécurisée des secrets} : déploiement de HashiCorp Vault afin de stocker et distribuer les informations sensibles.
    \item \textbf{Mise en place d'environnements distincts} : séparation claire entre les environnements de test, de pré-production et de production.
\end{itemize}

\section{Les besoins non fonctionnels}

Les besoins non fonctionnels définissent les critères de qualité que la solution doit respecter :

\begin{itemize}
    \item \textbf{Haute disponibilité} : assurer une disponibilité supérieure à 99,9\% des services critiques.
    \item \textbf{Sécurité renforcée} : garantir la protection des données sensibles et la conformité aux standards de sécurité.
    \item \textbf{Scalabilité} : permettre l'évolution de l'infrastructure en fonction de la croissance de l'activité et de l'augmentation du nombre de clients.
    \item \textbf{Maintenabilité} : faciliter les mises à jour, les correctifs et l'évolution des configurations.
    \item \textbf{Traçabilité et auditabilité} : conserver l'historique des changements et des déploiements.
    \item \textbf{Réduction du temps de mise en production} : diminuer les délais de déploiement des nouvelles fonctionnalités.
\end{itemize}
