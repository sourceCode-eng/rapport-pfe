Texts can be \textit{italics} and \textbf{bold}. Figures can be referenced (see Figure \ref{fig:javalogo} for an example). References can be cited; they are defined in the \textit{references.bib} file.\cite{example-reference} 

The \textit{Bibliography}, \textit{List of Figures} and \textit{List of Tables} are all automatically generated and references will be updated automatically.

If you've defined a citation but are not referencing it, it will not appear in the \textit{Bibliography}.

This is a list
\begin{itemize}
    \item item a
    \item item b
    \item ...
\end{itemize}

This is an enumerated list:
\begin{enumerate}
    \item item x
    \item item y
    \item ...
\end{enumerate}

\begin{figure}[ht]
    \centering
    \includegraphics[width=.5\textwidth]{figures/java.png}
    \caption{\textit{An image of Java logo.}}
    \label{fig:javalogo}
\end{figure}

A table with three columns can be seen in Table \ref{tab:data_table}.
\begin{longtable}{|p{0,5cm}|p{10cm}|p{3cm}|}
	\caption{\it{A table of some data}}
	\label{tab:data_table}\\ \hline
	\textbf{Nr} &  \textbf{Interests} & \textbf{Importance}  \\
	\hline
	\endfirsthead
	\multicolumn{3}{l}%
	{\tablename\ \thetable\ -- \textit{Continues...}} \\
	\hline
	\textbf{Nr} &  \textbf{Interests} & \textbf{Importance}  \\
	\hline
	\endhead
	\hline \multicolumn{3}{l}{\textit{Continues...}} \\
	\endfoot
	\hline
	\endlastfoot
1 & Big Data & High\\ \hline
2 & Data industry & Middle\\ \hline
3 & Artificial Intelligence & Low\\ \hline

\end{longtable}

We can use variables set in the \textit{main.tex} file to render values like our title (\doctitle) or supervisor names (\textbf{Supervisor}: \supervisor, \textbf{Co-supervisor}: \cosupervisor{}).