\section{Conclusion générale}

Le projet présenté dans ce mémoire s’inscrit dans un contexte d’évolution rapide des besoins en automatisation, en sécurité et en observabilité des infrastructures informatiques. Partant d’un environnement initial marqué par une forte hétérogénéité et des processus majoritairement manuels, l’objectif principal était de mettre en place une architecture moderne, automatisée et fiable, afin de soutenir la croissance et la performance de l’entreprise Oneex.

Les travaux réalisés ont permis d’atteindre ces objectifs à travers l’intégration d’outils et de pratiques DevOps éprouvés. L’infrastructure virtualisée, combinée à l’orchestration des conteneurs, à la gestion centralisée des secrets et à la mise en place d’un monitoring complet, constitue un socle robuste et évolutif.

Ce projet apporte une valeur ajoutée importante : une réduction significative du temps de déploiement, une amélioration de la sécurité des systèmes et une meilleure visibilité sur l’état des services. Ces bénéfices contribuent directement à renforcer la compétitivité et la capacité d’innovation de l’entreprise.

\section{Les limites du projet}

Malgré les résultats atteints, certaines limites ont été identifiées au cours du projet :

\begin{itemize}
	\item La complexité de la montée en compétence sur certaines technologies, notamment Vault et Argo CD, a nécessité un temps d’apprentissage important.
	\item Le projet a été limité par la disponibilité des ressources matérielles et le temps alloué, ce qui a retardé certaines phases de validation.
	\item La documentation des procédures n’a pas pu être finalisée de manière exhaustive dans les délais impartis.
	\item Certaines optimisations de performance et de sécurité (par exemple l’automatisation complète des sauvegardes chiffrées) n’ont pas pu être mises en œuvre.
\end{itemize}

\section{Les améliorations possibles}

Plusieurs pistes d’amélioration pourront être envisagées à l’avenir :

\begin{itemize}
	\item Compléter et enrichir la documentation technique et utilisateur.
	\item Renforcer la scalabilité du cluster Kubernetes pour accueillir de nouvelles applications.
	\item Intégrer des tests automatisés de sécurité (scans de vulnérabilités, compliance).
	\item Mettre en place des alertes proactives plus fines, basées sur des seuils dynamiques.
	\item Continuer la formation des équipes sur les outils mis en place afin de favoriser l’adoption.
\end{itemize}

Ces évolutions contribueront à renforcer encore la robustesse et la maturité de l’infrastructure.

\section{Les enseignements personnels}

Ce projet a été particulièrement riche en apprentissages, tant sur le plan technique que sur le plan humain.

Sur le plan technique, il m’a permis d’acquérir des compétences solides dans la mise en œuvre d’infrastructures automatisées, en explorant des outils variés tels que Terraform, Ansible, Kubernetes, Vault et Prometheus. J’ai également pu mieux comprendre les enjeux liés à la sécurité et à la haute disponibilité des services.

Sur le plan organisationnel, ce projet m’a appris à planifier des tâches complexes, à prioriser les actions et à collaborer efficacement avec les équipes internes. La nécessité de documenter chaque étape et de structurer les livrables a renforcé ma rigueur et ma capacité à travailler de manière autonome.

Enfin, cette expérience a confirmé mon intérêt pour le domaine du DevOps et de l’automatisation, et m’a donné envie de continuer à développer ces compétences dans un cadre professionnel.

% Si tu souhaites, tu peux ajouter des remerciements ici :
%\section*{Remerciements}
%\addcontentsline{toc}{section}{Remerciements}
%Je tiens à remercier l'ensemble des équipes d'Oneex pour leur accueil et leur soutien, ainsi que mon tuteur de stage pour ses conseils précieux tout au long de ce projet.

\section{Conclusion}

Ce projet de mise en place d'une infrastructure automatisée et sécurisée pour Oneex a permis de répondre à des enjeux à la fois opérationnels et stratégiques. En s’appuyant sur des outils modernes et des pratiques DevOps éprouvées, il a été possible d’améliorer significativement la fiabilité, la sécurité et la rapidité des déploiements.

L’architecture mise en œuvre offre une base solide et évolutive pour le développement futur des services, tout en assurant une gestion centralisée et sécurisée des configurations et des secrets. Les équipes disposent désormais d’un environnement cohérent, scalable et maintenable, capable de s’adapter aux besoins croissants de l’entreprise.

Ce mémoire a présenté de manière détaillée les différentes étapes de conception et de réalisation de cette solution, ainsi que les résultats obtenus. Il met en lumière l’importance de l’automatisation, de la sécurité et de l’observabilité dans la gestion d’infrastructures modernes, et souligne la nécessité d’adopter une approche proactive en matière de gouvernance technique.

Il convient de rappeler la distinction entre deux approches fondamentales en matière de sécurité :
\begin{itemize}
	\item \textbf{La sécurité par design} (\emph{security by design}) consiste à intégrer les mécanismes de protection dès la conception des systèmes, en anticipant les menaces et en appliquant systématiquement des principes comme le moindre privilège, la segmentation et la défense en profondeur.
	\item \textbf{La sécurité par obscurité} (\emph{security by obscurity}) repose uniquement sur la dissimulation des détails techniques (par exemple, masquer les configurations ou ne pas documenter les processus). Bien qu’elle puisse constituer une mesure complémentaire, elle ne peut en aucun cas se substituer à une politique de sécurité robuste et vérifiable.
\end{itemize}

Enfin, il est essentiel de continuer à investir dans la formation des équipes, l’évolution des outils et la diffusion des bonnes pratiques pour garantir la pérennité et la sécurité de l’infrastructure.

\subsection{Perspectives d’évolution}

L’initiative \textbf{Infra\_v2} ouvre la voie à plusieurs axes d’amélioration visant à renforcer la performance, la fiabilité et l’efficience des systèmes :

\begin{itemize}
	\item \textbf{Renforcement de l’automatisation} : intégrer de nouveaux processus automatisés, notamment les tests d’intégration, les vérifications de sécurité et les audits de conformité directement dans les pipelines CI/CD.
	\item \textbf{Observabilité avancée} : déployer une solution de télémétrie unifiée (par exemple OpenTelemetry) afin de collecter métriques, logs et traces dans un format standardisé, facilitant l’analyse et le diagnostic en temps réel.
	\item \textbf{Scalabilité horizontale} : permettre l’ajout dynamique de nœuds Kubernetes en fonction de la charge et de l’évolution des besoins applicatifs.
	\item \textbf{Sécurité renforcée} : généraliser le chiffrement des communications internes, la rotation automatique des secrets et l’application stricte du principe du moindre privilège.
	\item \textbf{Standardisation des workflows} : promouvoir l’adoption systématique des principes GitOps et l’harmonisation des pratiques de déploiement au sein de toutes les équipes.
	\item \textbf{Optimisation des coûts} : mettre en place des mécanismes de scaling automatique et d’analyse des consommations pour ajuster les ressources en fonction de l’activité et réduire les coûts d’infrastructure.
\end{itemize}

Ces perspectives s’inscrivent dans une démarche continue d’amélioration et d’industrialisation des processus techniques.

\subsection{Bilan technique et organisationnel}

L’expérience acquise dans ce projet a permis de mettre en évidence plusieurs points forts et points faibles, tant sur le plan technique qu’organisationnel.

\paragraph{Points forts techniques}
\begin{itemize}
	\item Infrastructure déclarative et versionnée, garantissant la reproductibilité et la traçabilité des configurations.
	\item Automatisation complète du cycle de vie des environnements (provisionnement, configuration, déploiement).
	\item Haute disponibilité native grâce à l’orchestration Kubernetes et au découplage des composants.
	\item Possibilité de rollback rapide et maîtrisé en cas d’incident.
	\item Intégration transparente d’un système centralisé de gestion des secrets.
\end{itemize}

\paragraph{Points faibles techniques}
\begin{itemize}
	\item Courbe d’apprentissage élevée pour la maîtrise et l’exploitation de l’ensemble des outils.
	\item Complexité accrue nécessitant une veille technologique constante et un maintien de compétences soutenu.
	\item Dépendance forte à l’intégrité des systèmes d’orchestration (GitOps, API Kubernetes), dont une indisponibilité peut impacter la production.
\end{itemize}

\paragraph{Points forts organisationnels}
\begin{itemize}
	\item Processus standardisés réduisant le risque d’erreurs humaines et améliorant la qualité globale des déploiements.
	\item Visibilité et transparence des configurations grâce au versionnement et à la centralisation.
	\item Accélération notable des cycles de livraison et meilleure réactivité des équipes.
	\item Renforcement de la collaboration inter-équipes via des workflows communs et partagés.
\end{itemize}

\paragraph{Points faibles organisationnels}
\begin{itemize}
	\item Nécessité d’une conduite du changement approfondie pour faire adopter les nouveaux outils et méthodologies.
	\item Risque de silos de compétences si la montée en compétence n’est pas homogène.
	\item Temps initial d’implémentation important pour structurer et standardiser l’ensemble des pipelines et des pratiques.
\end{itemize}

En synthèse, cette démarche constitue un socle technologique solide et évolutif, posant les bases d’un système plus résilient et sécurisé. Elle ouvre de nombreuses opportunités d’optimisation et d’innovation à moyen terme.
