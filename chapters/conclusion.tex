\section{Conclusion générale}

Le projet présenté dans ce mémoire s’inscrit dans un contexte d’évolution rapide des besoins en automatisation, en sécurité et en observabilité des infrastructures informatiques. Partant d’un environnement initial marqué par une forte hétérogénéité et des processus majoritairement manuels, l’objectif principal était de mettre en place une architecture moderne, automatisée et fiable, afin de soutenir la croissance et la performance de l’entreprise Oneex.

Les travaux réalisés ont permis d’atteindre ces objectifs à travers l’intégration d’outils et de pratiques DevOps éprouvés. L’infrastructure virtualisée, combinée à l’orchestration des conteneurs, à la gestion centralisée des secrets et à la mise en place d’un monitoring complet, constitue un socle robuste et évolutif.

Ce projet apporte une valeur ajoutée importante : une réduction significative du temps de déploiement, une amélioration de la sécurité des systèmes et une meilleure visibilité sur l’état des services. Ces bénéfices contribuent directement à renforcer la compétitivité et la capacité d’innovation de l’entreprise.

\section{Les limites du projet}

Malgré les résultats atteints, certaines limites ont été identifiées au cours du projet :

\begin{itemize}
	\item La complexité de la montée en compétence sur certaines technologies, notamment Vault et Argo CD, a nécessité un temps d’apprentissage important.
	\item Le projet a été limité par la disponibilité des ressources matérielles et le temps alloué, ce qui a retardé certaines phases de validation.
	\item La documentation des procédures n’a pas pu être finalisée de manière exhaustive dans les délais impartis.
	\item Certaines optimisations de performance et de sécurité (par exemple l’automatisation complète des sauvegardes chiffrées) n’ont pas pu être mises en œuvre.
\end{itemize}

\section{Les améliorations possibles}

Plusieurs pistes d’amélioration pourront être envisagées à l’avenir :

\begin{itemize}
	\item Compléter et enrichir la documentation technique et utilisateur.
	\item Renforcer la scalabilité du cluster Kubernetes pour accueillir de nouvelles applications.
	\item Intégrer des tests automatisés de sécurité (scans de vulnérabilités, compliance).
	\item Mettre en place des alertes proactives plus fines, basées sur des seuils dynamiques.
	\item Continuer la formation des équipes sur les outils mis en place afin de favoriser l’adoption.
\end{itemize}

Ces évolutions contribueront à renforcer encore la robustesse et la maturité de l’infrastructure.

\section{Les enseignements personnels}

Ce projet a été particulièrement riche en apprentissages, tant sur le plan technique que sur le plan humain.

Sur le plan technique, il m’a permis d’acquérir des compétences solides dans la mise en œuvre d’infrastructures automatisées, en explorant des outils variés tels que Terraform, Ansible, Kubernetes, Vault et Prometheus. J’ai également pu mieux comprendre les enjeux liés à la sécurité et à la haute disponibilité des services.

Sur le plan organisationnel, ce projet m’a appris à planifier des tâches complexes, à prioriser les actions et à collaborer efficacement avec les équipes internes. La nécessité de documenter chaque étape et de structurer les livrables a renforcé ma rigueur et ma capacité à travailler de manière autonome.

Enfin, cette expérience a confirmé mon intérêt pour le domaine du DevOps et de l’automatisation, et m’a donné envie de continuer à développer ces compétences dans un cadre professionnel.

% Si tu souhaites, tu peux ajouter des remerciements ici :
%\section*{Remerciements}
%\addcontentsline{toc}{section}{Remerciements}
%Je tiens à remercier l'ensemble des équipes d'Oneex pour leur accueil et leur soutien, ainsi que mon tuteur de stage pour ses conseils précieux tout au long de ce projet.
