Face à l’essor des architectures distribuées et aux exigences croissantes en matière de sécurité, de fiabilité et de conformité, les infrastructures classiques montrent rapidement leurs limites. Ce projet s’est attaché à repenser en profondeur la manière de concevoir et d’opérer une infrastructure informatique, dans un contexte réel au sein de l’entreprise Oneex.

L’approche adoptée ne s’est pas limitée à automatiser des déploiements ou à enchaîner des outils DevOps. Il s’est agi de construire une plateforme cohérente, robuste et évolutive, capable d’orchestrer des services critiques, de protéger les secrets, de tracer chaque changement et d’alerter intelligemment en cas d’incident.

Le système mis en place s’appuie sur une chaîne complète — de l’infrastructure virtuelle à la supervision applicative — avec un enchaînement réfléchi d’outils tels que Proxmox, Terraform, Ansible, Kubernetes, Vault, Argo CD et la stack d’observabilité Prometheus / Grafana / Loki / Tempo. Ces choix ne sont pas dictés par la mode, mais par leur complémentarité, leur adaptabilité et leur capacité à s’inscrire dans une démarche GitOps sécurisée.

Au-delà des aspects techniques, ce projet a permis d’explorer les vraies conditions de l’industrialisation : homogénéité entre environnements, résilience face aux erreurs humaines, visibilité sur les flux, et gouvernance des accès. Il constitue une base solide pour porter la transformation numérique de l’infrastructure de Oneex vers un modèle plus agile, contrôlé et résilient.